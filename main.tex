%%%%%%%%%%%%%%%%%%%%%%%%%%%%%%%%%%%%%%%%%%%%%%%%%%%%%%%%%%%%%%%%%%%%%%
% writeLaTeX Example: A quick guide to LaTeX
%
% Source: Dave Richeson (divisbyzero.com), Dickinson College
% 
% A one-size-fits-all LaTeX cheat sheet. Kept to two pages, so it 
% can be printed (double-sided) on one piece of paper
% 
% Feel free to distribute this example, but please keep the referral
% to divisbyzero.com
% 
%%%%%%%%%%%%%%%%%%%%%%%%%%%%%%%%%%%%%%%%%%%%%%%%%%%%%%%%%%%%%%%%%%%%%%
% How to use writeLaTeX: 
%
% You edit the source code here on the left, and the preview on the
% right shows you the result within a few seconds.
%
% Bookmark this page and share the URL with your co-authors. They can
% edit at the same time!
%
% You can upload figures, bibliographies, custom classes and
% styles using the files menu.
%
% If you're new to LaTeX, the wikibook is a great place to start:
% http://en.wikibooks.org/wiki/LaTeX
%
%%%%%%%%%%%%%%%%%%%%%%%%%%%%%%%%%%%%%%%%%%%%%%%%%%%%%%%%%%%%%%%%%%%%%%

\documentclass[dvipdfmx,a4paper,10pt,landscape]{article}

\usepackage{amsmath}
\usepackage{stix2}

\usepackage{multicol,multirow}
\usepackage{ifthen}
\usepackage[landscape]{geometry}
\usepackage[colorlinks=true,citecolor=blue,linkcolor=blue]{hyperref}

\ifthenelse{\lengthtest { \paperwidth = 11in}}
    { \geometry{top=.5in,left=.5in,right=.5in,bottom=.5in} }
	{\ifthenelse{ \lengthtest{ \paperwidth = 297mm}}
		{\geometry{top=1cm,left=1cm,right=1cm,bottom=1cm} }
		{\geometry{top=1cm,left=1cm,right=1cm,bottom=1cm} }
	}
\pagestyle{empty}
\makeatletter
\renewcommand{\section}{\@startsection{section}{1}{0mm}%
                                {-1ex plus -.5ex minus -.2ex}%
                                {0.5ex plus .2ex}%x
                                {\normalfont\large\bfseries}}
\renewcommand{\subsection}{\@startsection{subsection}{2}{0mm}%
                                {-1explus -.5ex minus -.2ex}%
                                {0.5ex plus .2ex}%
                                {\normalfont\normalsize\bfseries}}
\renewcommand{\subsubsection}{\@startsection{subsubsection}{3}{0mm}%
                                {-1ex plus -.5ex minus -.2ex}%
                                {1ex plus .2ex}%
                                {\normalfont\small\bfseries}}
\makeatother
\setcounter{secnumdepth}{0}
\setlength{\parindent}{0pt}
\setlength{\parskip}{0pt plus 0.5ex}

\DeclareMathOperator{\sinc}{sinc}
\DeclareMathOperator{\rect}{rect}
% -----------------------------------------------------------------------

\begin{document}

\raggedright
\footnotesize

\begin{center}
    \Large{\textbf{EE4704 Cheat Sheet}} \\
\end{center}
\begin{multicols}{3}
    \section{Fourier Transform (A)}
    \subsection{Basics}
    \subsubsection{Delta function}
    1D:\\
    \begin{gather*}
        \delta(x)=
        \begin{cases}
            0      & x \neq 0 \\
            \infty & x = 0
        \end{cases} \\
        f(x)\times\delta(x-a)=f(a)\delta(x-a) \\
        \int_{-\infty}^{\infty}f(x)\delta(x-a)dx=f(a)
    \end{gather*}
    2D: \\
    \begin{gather*}
        \delta(x,y)=
        \begin{cases}
            0      & x,y \neq 0 \\
            \infty & x,y = 0
        \end{cases} \\
        f(x,y)\delta(x-a,y-b)=f(a,b)\delta(x-a,y-b) \\
        \iint_{-\infty}^{+\infty}f(x,y)\delta(x-a,y-b)dxdy=f(a,b)
    \end{gather*}
    \subsubsection{Rectangle function}
    \begin{align*}
        \rect(x)   & =
        \begin{cases}
            1 & |x|\leq\frac{1}{2} \\
            0 & \text{otherwise}
        \end{cases} \\
        \rect(x,y) & =
        \begin{cases}
            1 & |x|,|y|\leq\frac{1}{2} \\
            0 & \text{otherwise}
        \end{cases}
    \end{align*}
    \subsubsection{Sinc function}
    \begin{align*}
        \sinc(x)   & =\frac{sin(\pi x)}{\pi x}                               \\
        \sinc(x,y) & =\frac{sin(\pi x)}{\pi x}\times\frac{sin(\pi y)}{\pi y} \\
                   & = \sinc(x)\sinc(y)
    \end{align*}

    \subsection{2D Fourier Transform}
    \begin{equation*}
        F(u,v)=\iint_{-\infty}^{+\infty}f(x,y)\exp[-j2\pi(ux+vy)]dxdy
    \end{equation*}
    Inverse:
    \begin{equation*}
        f(x,y)=\iint_{-\infty}^{+\infty}F(u,v)\exp[j2\pi(ux+vy)]dudv
    \end{equation*}
    Note
    \begin{align*}
        \exp[j2\pi(ux+vy)]  & =\cos[2\pi(ux+vy)]+j\sin[2\pi(ux+vy)] \\
        \exp[-j2\pi(ux+vy)] & =\cos[2\pi(ux+vy)]-j\sin[2\pi(ux+vy)]
    \end{align*}
    In general, the Fourier transform is complex:
    \begin{equation*}
        F(u,v)\equiv F_R(u,v)+jF_I(u,v)\equiv|F(u,v)|\exp[j\phi(u,v)]
    \end{equation*}
    with real part $F_R(u,v)$ and an imaginary part $F_I(u,v)$:
    \begin{center}
        \begin{tabular} {lll}
            Fourier spectrum: & $|F(u,v)|$  & $=[F_R^2(u,v)+F_I^2(u,v)]^{1/2}$ \\
            Phase spectrum:   & $\phi(u,v)$ & $=\tan^{-1}[F_I(u,v)/F_R(u,v)]$  \\
            Power spectrum:   & $P(u,v)$    & $=|F(u,v)|^2=R^2(u,v)+I^2(u,v)$
        \end{tabular}
    \end{center}
    Useful transform pairs:
    \begin{center}
        \begin{tabular}{ll}
            \hline
            Signal                     & Fourier transform                          \\
            \hline
            1                          & $\delta(u,v)$                              \\
            $\delta(x,y)$              & $1$                                        \\
            $\rect(x,y)$               & $\sinc(u,v)$                               \\
            $\sinc(x,y)$               & $\rect(u,v)$                               \\
            $e^{j2\pi(x+y)}$           & $\delta(u-1,v-1)$                          \\
            $e^{-(x^2+y^2)/2\sigma^2}$ & $2\pi\sigma^2e^{-2\pi^2\sigma^2(u^2+v^2)}$
        \end{tabular}
    \end{center}

    \subsection{Properties of 2D Continuous Transformation}
    \subsubsection{Linearity}
    \begin{equation*}
        \mathcal{F}\{af_1(x,y)+bf_2(x,y)\}=aF_1(u,v)+bF_2(u,v)
    \end{equation*}
    where $a$ and $b$ are constants.
    \subsubsection{Conjugate Symmetry}
    \begin{gather*}
        \mathcal{F}\{f^*(x,y)\}=F^*(-u,-v) \\
        |F(u,v)|=|F(-u,-v)|
    \end{gather*}
    where $^*$ denotes the complex conjugation of a variable,
    and $|F(u,v)|$ is symmetrical about the origin.
    \subsubsection{Translation}
    \begin{gather*}
        \mathcal{F}\{f(x-a,y-b)\}=F(u,v)\exp[-j2\pi(au+bv)] \\
        |\mathcal{F}\{f(x-a,y-b)\}|=|\mathcal{F}\{f(x,y)\}|
    \end{gather*}
    a spatial shift does not affect the magnitude of the transform.
    \subsubsection{Scaling}
    \begin{equation*}
        \mathcal{F}\{f(ax,by)\}=\frac{1}{|ab|}F(u/a,v/b)
    \end{equation*}
    \subsubsection{Parseval's Theorem}
    \begin{equation*}
        \iint_{-\infty}^{+\infty}|f(x,y)|^2dxdy=\iint_{-\infty}^{+\infty}|F(u,v)|^2 dudv
    \end{equation*}
    \subsubsection{Convolution}
    \begin{gather*}
        \mathcal{F}\{f(x,y)*h(x,y)\}=F(u,v)H(u,v)\\
        \mathcal{F}\{f(x,y)h(x,y)\}=F(u,v)*H(u,v)
    \end{gather*}
    \subsubsection{Rotation}
    \begin{equation*}
        x=r\cos\theta\quad y=r\sin\theta\quad u=\omega\cos\phi\quad v=\omega\sin\phi
    \end{equation*}
    then $f(x,y)$ and $F(u,v)$ become $f(r,\theta)$ and $F(\omega,\phi)$ and
    \begin{equation}
        f(r,\theta+\theta_0) \iff F(\omega,\phi+\theta_0)
    \end{equation}

\end{multicols}

\end{document}
